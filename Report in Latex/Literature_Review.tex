%%%%%%%%%%%%%%%%%%%%%%%%%%%%%%%%%%%%
\chapter{Literature Review} 
\label{chap:Literature_Review}
%%%%%%%%%%%%%%%%%%%%%%%%%%%%%%%%%%%%

Use your literature review to help the reader to understand the value and the interest in your project.  You should look for related works already published that either support the merit of your project, or provide the background understanding/information to make your new claims.  Try to avoid writing a "catalogue" of related works (e.g this would have little of your own insight added).  Instead, describe to the reader why the related work is interesting or relevant to your own work.  What did they achieve?  What did they overlook?  It is highly recommend you finish your Literature Review with a final subsection "Summary", where you may wish to formulate highly specified research questions or hypotheses, or assert the need for your Research Methodology (next chapter).  

introduction
literature review
implementation
research methodology
results




\section{This is a section}
\subsection{This is a subsection}
\subsubsection{This is subsubsection}

%%%%%%%%%%%%%%%%%%%%%%%%%%%%%%%%%%%%
% Figure with subfigures
%%%%%%%%%%%%%%%%%%%%%%%%%%%%%%%%%%%%

\begin{figure}[htb]
\centering
\begin{subfigure}[t]{.5\textwidth}
  \centering
  \includegraphics[height=4.5cm]{figures/Robot_1.jpg}
  \caption{\label{fig:left_robot} This is a robot.}
  \label{fig:theoretical}
\end{subfigure}%
\begin{subfigure}[t]{.5\textwidth}
  \centering
  \includegraphics[height=4.5cm]{figures/Robot_2.jpg}
  \caption{\label{fig:right_robot} This another robot.}
  \label{fig:practical}
\end{subfigure}
\caption{\label{fig:two_robots} These are two robots}
\label{fig:test}
\end{figure}

For example, \cite{Robots2020} discusses the two robots depicted in Figure \ref{fig:two_robots}. There is a robot in Figure \ref{fig:left_robot} and another robot in Figure \ref{fig:right_robot}.

